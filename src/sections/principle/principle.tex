\documentclass[../../../main]{subfiles}
\begin{document}

\section{原理}

\subsection{ホール効果}
図\ref{fig:hall-effect}のように、板状または棒状の半導体に対し、
$x$方向に電流$I_x$を流し、$z$方向に磁場$B_z$を印加する。
半導体中を移動するキャリアは、ローレンツ力により、$y-$方向に偏向される。
この偏りによって電場$E_H$が生じ、キャリアが電子の場合$y-$方向、正孔の場合$y+$方向を向きである。
このホール電場から受ける力とローレンツ力が釣り合うときが定常状態で、
このときの上面と下面の電位差をホール電圧$V_H$といい、
\begin{align}\label{eq:hall-voltage}
    I_x B_z x & = q (n x y t) \dfrac{V_H}{y} \nonumber \\
    V_H       & = \dfrac{I_x B_z}{q n t}               \\
              & = R_H \dfrac{I_x B_z}{t}
\end{align}
である。
ただし、$x, y, t$はそれぞれ$x, y, z$方向の半導体の長さ、$q$は電荷、$n$はキャリア数密度である。
また、
\begin{equation}\label{eq:hall-coefficient}
    R_H = \dfrac{1}{q n}
\end{equation}
をホール係数という。
\ref{eq:hall-coefficient}式より、キャリア数密度は、
\begin{equation}\label{eq:carrier-density}
    n = \dfrac{1}{q R_H}
\end{equation}
とかける。
移動度$\mu$は、
\begin{equation}\label{eq:mobility}
    \sigma = \dfrac{1}{\rho} = q n \mu
\end{equation}
と表され、\ref{eq:carrier-density}式を代入すると、
\begin{equation}\label{eq:mobility-1}
    \mu = \sigma R_H
\end{equation}
と表される。



\end{document}
